\documentclass{article}

%% preamble
\usepackage{hyperref}
\usepackage{verbatim}
\usepackage{color}
\usepackage{graphicx}
\usepackage{amsmath}
\usepackage{amsfonts}
\usepackage{cancel}

\topmargin 0pt
\advance \topmargin by -\headheight
\advance \topmargin by -\headsep
\textheight 9.in
\oddsidemargin 0pt
\evensidemargin \oddsidemargin
\marginparwidth 0.5in
\textwidth 6.5in
\newcommand{\myhrule}{ \begin{center}\rule{.9\linewidth}{.25mm}\end{center} }
\definecolor{darkgray}{rgb}{0.95,0.95,0.95}
\definecolor{heavygray}{rgb}{0.05,0.05,0.05}
\definecolor{foo}{rgb}{.8,0,.8}
\newcommand{\pad}{\vspace{8pt}\noindent}
\newcommand{\red}[1]{{\color{red}#1\color{black}}}
\newcommand{\myhref}[2]{\href{#1}{\color{foo}\underline{#2}\color{black}}}

\newcommand{\Mod}[1]{\ \mathrm{mod}\ #1}


\begin{document}

\title{CMDA 3634 Spring 2018 Homework 03}

%% change this to your name
\author{Pim Silpacharn}
\vspace{-64pt}\maketitle
\begin{center}\underline{You must complete the following task by 5pm on Tuesday 03/27/18.}\end{center}
Your write up for this homework should be presented in a {\LaTeX} formatted PDF document. You may copy the \LaTeX{} used to prepare this report as follows

\begin{enumerate}
\item Click on this  \myhref{https://www.sharelatex.com/read/tkfwfqfjqqgs}{link} 
\item Click on Menu/Copy Project.
\item Modify the HW03.tex document to respond to the following questions. 
\item Remember: click the Recompile button to rebuild the document when you have made edits.
\item Remember: Change the author. 

\end{enumerate}

\pad \emph{Each student} must individually upload the following files to the CMDA 3634 Canvas page at \myhref{https://canvas.vt.edu}{https://canvas.vt.edu}

\begin{enumerate}
\item \verb|firstnameLastnameHW03.tex| {\LaTeX} file.
\item Any figure files to be included by \verb|firstnameLastnameHW03.tex| file.
\item \verb|firstnameLastnameHW03.pdf| PDF file.
\end{enumerate}

\noindent In addition, all source code must be submitted to an online git repository as follows:
\begin{enumerate}
    \item While on the webpage for you git repository, go to Settings $\rightarrow$ Collaborators.
    \item Add nchalmer@vt.edu and zjiaqi@vt.edu as collaborators. 
    \item Make a folder named HW03 in you repository and store all relevant source files to this assignment in this folder. Ensure your assignment files compile with {\verb make }.
\end{enumerate}

\pad You must complete this assignment on your own.

\vspace{16pt}
\begin{center}
\underline{\bf 160 points will be awarded for a successful completion.}
\vspace{8pt}\underline{\bf Extra credit will be awarded as appropriate.}
\end{center}

\newpage

\vspace*{1em}
\noindent{\bf Q0.1}(30 points) 30 points is awarded for attending class on March 13th and submitting ``Computing pi in parallel using OpenMP'' code to GitHub. All code and any plotting results should be saved to your GitHub repository. Any additional comments can be included here in this homework. 
\vspace*{1em}

\vspace*{1em}
\noindent{\bf Q0.2}(30 points) 30 points is awarded for attending class on March 15th and submitting ``adding threading to a Mandelbrot set generator'' code to GitHub. All code and any plotting results should be saved to your GitHub repository. Any additional comments can be included here in this homework. 

\vspace*{1em}

\section*{ElGamal Public-key Cyptography}

In the previous assignment we added some safe modular products and exponentiation functions, and used these to program functions which can identify very large prime numbers $p$ and even easily find generators of $\mathbb{Z}_p$ for some specific primes. With these tools, we can now reliably setup an ElGamal cryptographic system. In this assignment, we will implement the ElGamal encryption and decryption algorithms and begin exploring how to use parallelism if we wish to determine the secret key used in an ElGamal cryptographic system.

\subsection*{Part 1}
In the {\verb HW03 } folder you will find two folders, {\verb Part1 } and {\verb Part2 }. This section concerns the contents of the {\verb Part1 } folder. 

\pad {\bf Setting up an ElGamal Cryptographic System}:
Recall that in the description of the ElGamal cryptographic system in Homework 1 we considered three different people, Alice, Bob, and Eve. In the system, Alice sets up an ElGamal cryptographic system and shares her \emph{public-key} which consists of the prime number $p$, generator $g$, and number $h$. She also secretly selects a number $x$, keeping this number private. Using the public-key, Bob is able to send messages that only Alice can read by encrypting his messages $m$ using $p$, $g$, and $h$ in such a way that only Alice is able to decrypt the resulting cyphertext. 

Let us consider a single program which can emulate this process. In the source file {\verb main.c } we have begun a program which must be run with MPI using at least 2 ranks. We will label rank 0 as Alice and rank 1 as Bob.

\vspace*{1em}
\noindent{\bf Q1.1}(5 points) The program should begin with Alice setting up her ElGamal cryptographic system. Since this the same process each time, we have added the function {\verb setupElGamal } in the {\verb functions.c } file. Write this function so it calculates and returns the necessary values for an ElGamal cryptographic system.

\vspace*{1em}
\noindent{\bf Q1.2}(5 points) Alter the {\verb main } function so that only Alice inputs a bit length from the user and sets up an ElGamal cryptographic system. 
\vspace*{1em}

\vspace*{1em}
\noindent{\bf Q1.3}(5 points) Use MPI to broadcast the values which are part of Alice's public-key to all MPI processes. 
\vspace*{1em}

\pad {\bf Encryption and Decryption}:
Once the ElGamal system is set up, the crucial steps that remain are the encryption and decryption functions. The {\verb main } function currently creates an array of {\verb Nmessages=5 } messages, encrypts and then decrypts them, outputting their values along the way. 

\vspace*{1em}
\noindent{\bf Q2.1}(10 points) In the file {\verb functions.c } we have begun implementing the {\verb ElGamalEncrypt } function. Use the description of the ElGamal encryption process from Homework 1 to complete this function.
\vspace*{1em}

\vspace*{1em}
\noindent{\bf Q2.2}(10 points) In the file {\verb functions.c } we have begun implementing the {\verb ElGamalDecrypt } function. Use the description of the ElGamal decryption process from Homework 1 to complete this function. 
\vspace*{1em}

\vspace*{1em}
\noindent{\bf Q2.3}(15 points) Alter the {\verb main } function so that only Bob populates the message array and encrypts it. After encryption, use MPI to send the encrypted messages to Alice. Alice should then receive the messages and decrypt them. Run your program and confirm that Alice obtains Bob's original messages after decryption. 
\vspace*{1em}

\subsection*{Part 2}
In the {\verb HW03 } folder you will find two folders, {\verb Part1 } and {\verb Part2 }. This section concerns the contents of the {\verb Part2 } folder. Before completing this section, copy the contents of your {\verb setupElGamal }, {\verb ElGamalEncrypt }, and {\verb ElGamalDecrypt } functions completed above to {\verb functions.c } in this folder. 

\pad {\bf Cracking the ElGamal Cryptographic System}:
Once the messages $m$ are encrypted, is very difficult to decrypt them without knowledge of the random number $y$ used in the encryption algorithm, or the secret key $x$. In the case where we have many messages, finding each $y$ is not feasible and we must instead determine the secret key. This amounts to finding the number $x$ such that 
\[
h = g^x \Mod{p}.
\]
Trying to solve this equation by brute force (that is, literally trying every value $x$) is, luckily, embarrassingly parallel. In this part we will use MPI to accelerate the process of cracking the ElGamal cryptographic system by distributing the job of trying each value of $x$ amoungst many MPI processes.  
\vspace*{1em}

\noindent{\bf Q3.1}(5 points) Alter the {\verb main } function so that only rank 0 sets up the ElGamal cryptographic system and then broadcast the public key information to all MPI processes. 
\vspace*{1em}

\noindent{\bf Q3.2}(15 points) In the {\verb main } function, all MPI processes currently loop through all values of $x$. Partition this loop between all MPI processes by setting values for {\verb start } and {\verb end } so that all values of $x$ are tested, and the amount of work each MPI process performs is roughly equal. 
\vspace*{1em}

\pad {\bf Parallel performance}:

\noindent{\bf Q4.1}(10 points) What would be a good measure of the work performed inside the loop in your program's {\verb main } function? Use {\verb MPI_Wtime() } to compute the total time this loop takes to run inside your program and use your measurement of work to compute your program's throughput. Output both runtime and throughput quantities at the end of your program. 
\vspace*{1em}

\noindent{\bf Q4.2}(20 points) Using New River, run your program several times for $N = 1, 2, 4, 8, 12, 16,$ and 20 MPI processes. You may need to alter your main file to fix the number of bits $n$, rather than inputting it from the user. Plot the average runtime of your program vs $N$ for a variety of bit lengths. In a separate plot, show the average throughput numbers reported by your program. Give some comments on how this program's performance scales with the number of MPI processes. 
\vspace{5mm} \\
\textbf{Comments}: As the number of MPI processes increases, the runtime decreases. Increasing the bit size numbers increases the runtime. I noticed that the throughput quantities increased throughout but I am unsure I did this process correctly.
\vspace*{1em}


\vspace*{1em}
\noindent{\bf Bonus:}(20 points) In practice, we would like to stop searching for the secret key $x$ once it is found, rather than continuing the loop. Alter the loop in your {\verb main } function so that at regular intervals, say every {\verb Ninterval } iterations, you determine if the secret key has been found, and stop iterating if so. Remake your runtime and throughput plots in Q3.3. If you averaged over a large number of runs, how much faster would you expect this updated version of the code to run? Does the frequency at which you check if the secret has been found, i.e. the size of {\verb Ninterval } affect the performance? 

%\myhrule

%\begin{thebibliography}{9}
%\bibitem{gol} Wikipedia -- description of the Game of Life
%\url{https://en.wikipedia.org/wiki/Conway's_Game_of_Life}
 
%\end{thebibliography}


\end{document}